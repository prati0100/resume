%%%%%%%%%%%%%%%%% PREAMBLE %%%%%%%%%%%%%%%%%%%%%%%%%%%%
\documentclass[letterpaper,11pt,oneside]{article}
\usepackage[utf8]{inputenc}
\usepackage{setspace}
\usepackage[hidelinks]{hyperref}
\usepackage{tabularx}
\usepackage{ltablex}
\usepackage{changepage}
\usepackage{enumitem}

\usepackage[left=0.25in, right=0.5in, bottom=0.75in, top=0.75in]{geometry}

%Changes the page numbers - {arabic}=arabic numerals, {gobble}=no page numbers, {roman}=Roman numerals
\pagenumbering{gobble}

%%%%%%%%%%%%%%%%% END OF PREAMBLE %%%%%%%%%%%%%%%%%%%%%

\begin{document}

%%%%%%%%%%%%%%%%% NAME OF APPLICANT %%%%%%%%%%%%%%%%%%%

\noindent  \hspace{0.25in} \LARGE{\textbf{Pratyush Yadav}}  \\
\vspace{-2ex}
\hline
\normalsize

%%%%%%%%%%%%%%%%% CONTACT INFORMATION %%%%%%%%%%%%%%%%%

\begin{center}
\begin{tabular}{l l}
 Manipal Institute of Technology    & \hspace{1in} \href{mailto:me@yadavpratyush.com}{me@yadavpratyush.com} \\
 Manipal, Karnataka, India & \hspace{1in}  \href{https://yadavpratyush.com}{yadavpratyush.com}   \\
 & \hspace{1in} Phone: +91 9034438261 \\
\end{tabular}
\end{center}

\vspace{-1em}

%%%%%%%%%%%%%%%%% MAIN BODY %%%%%%%%%%%%%%%%%%%%%%%%%%%
% The main body is contained in a tabular environment. To move sections onto the next page, simply end the tabular environment and begin a new tabular environment.

\noindent \begin{tabularx}{\textwidth}{l X}
 \Large{Education} & \textbf{Manipal Institute of Technology} \hfill 2016-2020  \\
     & Bachelor of Technology \\
     & Computer Science and Engineering \\
     &  CGPA: 8.81\\
     & \\
 \Large{Experience}    & \textbf{Google Summer of Code 2018} \hfill April - August 2018 \\
    & The FreeBSD Foundation \\
    & Imported Xen grant-table bus\_dma(9) handlers from OpenBSD. \vspace{-1ex}
    \begin{itemize}[label={--}]
    \setlength\itemsep{-0.25em}
        \item Grant tables are a mechanism used to communicate between the Xen hypervisor VMs using shared pages.
        \item Wrote an implementation of the FreeBSD kernel's bus\_dma(9) interface integrating the grant table mechanism with the interface.
        \item Updated the Xen paravirtualized I/O drivers to use the new implementation, resulting in a more transparent integration with the rest of the OS.
    \end{itemize} \\
    & \textbf{FreeBSD} \hfill April 2018 - Present \\
    & Contributor \vspace{-1ex}
    \begin{itemize}[label={--}]
        \setlength\itemsep{-0.25em}
            \item Imported the Xen grant-table bus\_dma(9) handlers from OpenBSD (GSoC 2018).
            \item Multiple bug fixes to the Xen paravirtualized grant table, block device, and network device drivers.
            \item Run grant-table callbacks from a separate context to avoid the chances of a deadlock (WIP).
    \end{itemize} \\
    & \textbf{ACM Student Chapter Manipal} \hfill April 2018 -  Present \\
    & President \\
    & I organize and coordinate a team of about 40 people to host technical workshops, competitions and events. \\
    & \\
 \Large{Skills} & \textbf{Working knowledge:} \\
    & C, Operating Systems, FreeBSD Kernel \\
    & \\
    & \textbf{Past Experience:} \\
    & Java, Android App Development, Git, Xen device drivers \\
    & \\
 \Large{Projects} \vspace{-1.5ex} & \textbf{\href{https://github.com/prati0100/OS161}{OS/161}} \\
    \\
    & OS/161 is a teaching OS created by Harvard for their OS course assignments. \vspace{-0.75ex}
    \begin{itemize}[label={--}]
    \setlength\itemsep{-0.25em}
        \item Implemented address spaces and the virtual memory subsystem.
        \item Implemented process support and system calls like fork(), exec(), read(), write(), etc.
    \end{itemize} \\
    % \pagebreak
    \vspace{-2.5ex} &  \textbf{\href{https://github.com/prati0100/P2PChat}{P2PChat}} \\
    \vspace{-3ex} & \begin{adjustwidth}{}{}A simple terminal-based peer to peer chat application written in Java as a part of my Objected Oriented Programming class.\end{adjustwidth} \\
    & \\
    % TODO: This thing with vspaces is too hacky. Try to figure out a more elegent solution when you find the time.
    & \vspace{-3ex} \textbf{\href{https://github.com/prati0100/psh}{PSH}} \\
    & \vspace{-1.5ex} psh is a simple shell written in C as a fun learning exercise. It can:
    \begin{itemize}[label={--}]
    \setlength\itemsep{-0.25em}
        \item Execute any binary in the PATH with the ability to pass command line arguments.
        \item Add aliases for commands.
        \item Read from the .pshrc file on startup to set up aliases, etc.
    \end{itemize} \\
    & \vspace{-3ex}\textbf{\href{https://github.com/prati0100/EventManager}{Event Manager}} \\
    & \vspace{-2.5ex}\begin{adjustwidth}{}{}An event management app I created with a classmate as a part of our database systems class to demonstrate Android's SQLite support.\end{adjustwidth} \\
    & \\
\end{tabularx}

\centerline{Last updated 05 January, 2019}
\end{document}
